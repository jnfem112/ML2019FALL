\documentclass{article}

\usepackage{enumitem}
\setenumerate[1]{itemsep = 1pt , partopsep = 1pt , parsep = \parskip , topsep = 1pt}
\usepackage{amsmath}
\usepackage{amsfonts}
\usepackage{xeCJK}
\setCJKmainfont[AutoFakeBold = 2.5 , AutoFakeSlant = 0.4]{cwTeXKai}
\newcommand{\parallelsum}{\mathbin{\!/\mkern-5mu/\!}}
\usepackage{multirow}
\usepackage{makecell}
\usepackage{hyperref}
\hypersetup{colorlinks = true , linkcolor = blue , filecolor = magenta , urlcolor = blue}
\usepackage{graphicx}
\graphicspath{{image/}}
\usepackage{stmaryrd}

\title{Machine Learning - Final Project\\{\Large Domain Adaptation}}
\author{資工四 B05902021 徐祐謙\\資工四 B05902023 李澤諺\\資工四 B05902120 曾鈺婷}
\date{January 16, 2020}

\begin{document}

\maketitle

\noindent
{\bf \LARGE Introduction and Motivation}\\

本次final project的題目為:給定10個class的真實照片,試著用其訓練出一個model,以此將手繪的圖片進行分類。之所以選擇這個題目,是因為domain adaptation為transfer learning中一個很重要的問題:當我們想要處理的domain (稱為target domain)中沒有足夠多的data,但是在另一個domain (稱為source domain)中的data充足,可以讓我們在source domain中解決相同的問題,那我們就可以使用domain adaptation,透過在source domain中解決相同的問題,來幫助我們解決在target domain中的原問題。考慮到現實中蒐集data的困難度,我們不見得能在欲處理的target domain上蒐集到足夠多的data,不過網路上已有許多現成的大型dataset,其domain雖然可能會和我們想要處理的target domain不同,但是只要利用transfer learning中的domain adaptation,仍能幫助我們解決原本的問題,由此可見transfer learning在現實中的重要性。\\

\bigskip

\noindent
{\bf \LARGE Data Preprocessing/Feature Engineering}\\

\begin{enumerate}
    \item[(1)] 因為testing data為手繪的黑白圖片,圖片中僅有物體的輪廓,而training data則是真實物體的彩色照片,為了讓training data盡可能和testing data相似,我們對training data使用了Canny edge detection,將training data中物體的edge找出,如此一來training data便會在外觀上和testing data相似(如下圖),此時我們再用training data去訓練model,便有可能提高accuracy。\\
\end{enumerate}

\begin{center}
    \includegraphics[width=0.8\textwidth]{Canny_0.png}
\end{center}
\begin{center}
    \includegraphics[width=0.8\textwidth]{Canny_4.png}
\end{center}
\begin{center}
    \includegraphics[width=0.8\textwidth]{Canny_6.png}
\end{center}

\begin{enumerate}
    \item[] 由於training data中除了欲分類的物體外,還有背景或其它物體等等,其經過Canny edge detection後也會留下輪廓,為一種noise,其對model的訓練會造成影響,為了減少背景或不相關的物體所產生的輪廓,我們試著將Canny edge detection的threshold調大(cv2.Canny(image, 250, 300)),將大部分的輪廓去掉,以減少雜訊。\\
    \item[(2)] 由於training data的大小為 $32 \times 32$,而testing data的大小為 $28 \times 28$,因此我們有將testing data放大為 $32 \times 32$,再進行訓練與測試。\\
    \item[(3)] 我們有將training data和testing data的值皆除以255,使得pixel的值normalize到0和1之間。\\
    \item[(4)] 在訓練過程中,我們會將training data和testing data皆經過RandomAffine(10, translate = (0.1, 0.1), scale = (0.9, 1.1))和RandomHorizontalFlip(),以增加data的數量。\\
\end{enumerate}

\bigskip

\noindent
{\bf \LARGE Model Description}\\

我們所使用的model為MCD \href{https://github.com/mil-tokyo/MCD_DA}{[ref]},其架構為:前半部為一個feature extractor,其後接有兩個classifier。訓練過程中,每一個epoch皆分為三個步驟(如下圖所示):\\

\begin{enumerate}
    \item[(1)] 訓練feature extractor和兩個classifier,使得feature extractor要能從source data中找出夠好的feature,並且兩個classifier皆要能以此將source data做出正確的分類。\\
    \item[(2)] 在固定feature extractor的parameter之下,訓練兩個classifier,使得兩個classifier要在依然能正確分類source data的條件下,還要使得兩個classifier在target data上預測的distribution相差越多,增加兩個classifier的decision boundary之間的距離。\\
    \item[(3)] 在固定兩個classifier的parameter之下,訓練feature extractor,使得feature extractor在target data上所找出的feature,經由兩個classifier預測後的distribution相差越多。
\end{enumerate}

\begin{center}
    \includegraphics[width=\textwidth]{MCD_train.png}
\end{center}

在經過如此的訓練過程後,可以期望feature extractor在source domain和target domain上所找出的feature可以彼此align,如此一來只要classifier能夠將source data正確地分類,便也有可能將target data進行正確地分類。\\

接著,我們所實作的MCD架構如下:

\begin{center}
    \includegraphics[width=\textwidth]{MCD.png}
\end{center}

\begin{center}
    \begin{tabular}{|c|l|}
        \hline
        \multicolumn{2}{|c|}{MCD}\\
        \hline
        \multirow{14}*[-2.25em]{\makecell{feature\\extractor}} & \makecell[l]{Conv2d(1 , 64 , kernel\_size = (5 , 5) , stride = (1 , 1) ,\\padding = (2 , 2))}\\
        \cline{2-2}
        & BatchNorm2d(64)\\
        \cline{2-2}
        & ReLU()\\
        \cline{2-2}
        & \makecell[l]{MaxPool2d(kernel\_size = (3 , 3) , stride = (2 , 2) ,\\padding = (1 , 1))}\\
        \cline{2-2}
        & \makecell[l]{Conv2d(64 , 64 , kernel\_size = (5 , 5) , stride = (1 , 1) ,\\padding = (2 , 2))}\\
        \cline{2-2}
        & BatchNorm2d(64)\\
        \cline{2-2}
        & ReLU()\\
        \cline{2-2}
        & \makecell[l]{MaxPool2d(kernel\_size = (3 , 3) , stride = (2 , 2) ,\\padding = (1 , 1))}\\
        \cline{2-2}
        & \makecell[l]{Conv2d(64 , 128 , kernel\_size = (5 , 5) , stride = (1 , 1) ,\\padding = (2 , 2))}\\
        \cline{2-2}
        & BatchNorm2d(128)\\
        \cline{2-2}
        & ReLU()\\
        \cline{2-2}
        & Linear(8192 , 3072)\\
        \cline{2-2}
        & BatchNorm1d(3072)\\
        \cline{2-2}
        & ReLU()\\
        \hline
        \multirow{4}*{classifier} & Linear(3072 , 2048)\\
        \cline{2-2}
        & BatchNorm1d(2048)\\
        \cline{2-2}
        & ReLU()\\
        \cline{2-2}
        & Linear(2048 , 10)\\
        \hline
    \end{tabular}
\end{center}

我們使用了Adam訓練MCD,其中learning rate為0.00002,weight decay為0.0005,batch size為128,訓練了2000個epoch,以此得到MCD。\\

\bigskip

\noindent
{\bf \LARGE Experiment and Discussion}\\

除了使用MCD之外,我們也有試過將training data經過Canny edge detection處理之後直接給CNN進行訓練,以下為CNN和MCD在Kaggle上分別所得到最高的accuracy:

\begin{center}
    \begin{tabular}{|c|c|c|}
        \hline
        & Public & Private\\
        \hline
        CNN & 0.41526 & 0.42044\\
        \hline
        MCD & 0.80093 & 0.80145\\
        \hline
    \end{tabular}
\end{center}

由此可以看出直接使用CNN所得到的accuracy較差,其原因大致是因為雖然我們讓training data和testing data在外觀上盡可能地相似,但兩者之間在feature上的差異仍然很大,必須使用transfer learning去找出training data和testing data在feature上的相同與相
異之處,讓training data和testing data在feature上相似,才能讓model正確地將testing data進行分類。\\

\bigskip

\noindent
{\bf \LARGE Conclusion}\\

Transfer learning在現實中其實很常發生,經由本次的final project我們也對其學習到了不少,當source domain和
target domain有所差異時,僅僅只是使用data preprocessing讓兩者的外觀相似無法得到好的accuracy,必須使用transfer learning中的domain adaptation才能真正學到兩個domain之間feature上的相同與相異之處,進而做出正確的分類,而在前三名組別的分享中,皆有使用到pseudo-label (例如:若model在作測試時,認為某一筆testing data為某一個class的機率高於threshold,就將該class當作該筆testing data的pseudo-label,或是有組別將testing data進行cluster,分為1000個group,若model預測某一個group之中某一個class所佔的比例高於threshold,就將該class當作這個group之中所有testing data的pseudo-label),使用pseudo-label訓練CNN,可以得到更高的accuracy,我們認為其原因為:transfer learning在target domain上是完全沒有label的,對target domain是一無所知,而若我們相信pseudo-label的正確性,則此時我們在target domain上就有label,知道一部分的正確答案,就可以進行supervised learning,比其它的學習方法略勝一籌,由此可見pseudo-label的重要性。總之,經由本次的final project,我們學習到了transfer learning的基本流程,收穫著實頗豐。

\end{document}