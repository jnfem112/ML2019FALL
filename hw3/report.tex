\documentclass{article}

\usepackage{enumitem}
\setenumerate[1]{itemsep = 1pt , partopsep = 1pt , parsep = \parskip , topsep = 1pt}
\usepackage{multirow}
\usepackage{amsmath}
\usepackage{amsfonts}
\usepackage{xeCJK}
\setCJKmainfont[AutoFakeBold = 2.5 , AutoFakeSlant = 0.4]{cwTeXKai}
\newcommand{\parallelsum}{\mathbin{\!/\mkern-5mu/\!}}
\usepackage{mathtools}
\DeclarePairedDelimiter\ceil{\lceil}{\rceil}
\DeclarePairedDelimiter\floor{\lfloor}{\rfloor}
\usepackage{hyperref}
\hypersetup{colorlinks = true , linkcolor = blue , filecolor = magenta , urlcolor = blue}
\usepackage{graphicx}
\graphicspath{{image/}}
\usepackage[ruled,vlined,linesnumbered]{algorithm2e}

\title{Machine Learning - Homework 3}
\author{資工四 B05902023 李澤諺}
\date{November 11, 2019}

\begin{document}

\maketitle

\noindent
{\bf \LARGE Part 1. Programming Problem}\\

\noindent
{\bf 1. (1\%) 請說明這次使用的model架構,包含各層維度及連接方式。}\\

以下為我於本次作業中所實作的CNN架構:

\begin{center}
    \includegraphics[width=\textwidth]{CNN.png}
\end{center}

\begin{center}
    \begin{tabular}{|l|}
        \hline
        Conv2d(1 , 32 , kernel\_size = (3 , 3) , stride = (1 , 1) , padding = (2 , 2))\\
        \hline
        BatchNorm2d(32)\\
        \hline
        ReLU()\\
        \hline
        MaxPool2d(2)\\
        \hline
        Dropout2d()\\
        \hline
        Conv2d(32 , 64 , kernel\_size = (3 , 3) , stride = (1 , 1) , padding = (2 , 2))\\
        \hline
        BatchNorm2d(64)\\
        \hline
        ReLU()\\
        \hline
        MaxPool2d(2)\\
        \hline
        Dropout2d()\\
        \hline
        Conv2d(64 , 128 , kernel\_size = (3 , 3) , stride = (1 , 1) , padding = (2 , 2))\\
        \hline
        BatchNorm2d(128)\\
        \hline
        ReLU()\\
        \hline
        MaxPool2d(2)\\
        \hline
        Dropout2d()\\
        \hline
        Conv2d(128 , 256 , kernel\_size = (3 , 3) , stride = (1 , 1) , padding = (2 , 2))\\
        \hline
        BatchNorm2d(256)\\
        \hline
    \end{tabular}
\end{center}

\begin{center}
    \begin{tabular}{|l|}
        \hline
        ReLU()\\
        \hline
        MaxPool2d(2)\\
        \hline
        Dropout2d()\\
        \hline
        Conv2d(256 , 512 , kernel\_size = (3 , 3) , stride = (1 , 1) , padding = (2 , 2))\\
        \hline
        BatchNorm2d(512)\\
        \hline
        ReLU()\\
        \hline
        MaxPool2d(2)\\
        \hline
        Dropout2d()\\
        \hline
        Conv2d(512 , 1024 , kernel\_size = (3 , 3) , stride = (1 , 1) , padding = (2 , 2))\\
        \hline
        BatchNorm2d(1024)\\
        \hline
        ReLU()\\
        \hline
        MaxPool2d(2)\\
        \hline
        Dropout2d()\\
        \hline
        Linear(4096 , 100 , bias = True)\\
        \hline
        BatchNorm1d(100)\\
        \hline
        ReLU()\\
        \hline
        Dropout()\\
        \hline
        Linear(100 , 100 , bias = True)\\
        \hline
        BatchNorm1d(100)\\
        \hline
        ReLU()\\
        \hline
        Dropout()\\
        \hline
        Linear(100 , 7 , bias = True)\\
        \hline
    \end{tabular}
\end{center}

\bigskip

\noindent
{\bf 2. (1\%) 請附上model的training/validation history (loss and accuracy)。}\\

我隨機選出了2000張照片作為validation data,剩下的照片則作為training data,並且,我使用了Adam訓練CNN,其中learning rate為0.0001,batch size為1024,以此訓練了2000個epoch,所得到的loss和accuracy如下圖所示:

\begin{center}
    \includegraphics[width=\textwidth]{CNN_loss.png}\\
\end{center}

\begin{center}
    \includegraphics[width=\textwidth]{CNN_accuracy.png}\\
\end{center}

\noindent
{\bf 3. (1\%) 畫出confusion matrix分析哪些類別的圖片容易使model搞混,並簡單說明。 (ref:\href{https://en.wikipedia.org/wiki/Confusion_matrix}{https://en.wikipedia.org/wiki/Confusion\_matrix}})\\

\begin{center}
    \includegraphics[width=\textwidth]{confusion_matrix.png}\\
\end{center}

由confusion matrix可以看出,最容易辨識錯誤的情緒為將surprise辨識為fear,其次為將fear辨識為angry,再者為將fear辨識為sad以及將sad辨識為angry,我認為其原因為這些被辨識錯誤的情緒多為負面情緒,其在人的面部表情呈現上其實相差無多,連人們也經常會辨識錯誤,在電腦的表現上似乎也是如此。\\

\bigskip

\noindent
{\bf [關於第四及第五題]}\\
{\bf 可以使用簡單的3-layer CNN model [64, 128, 512] 進行實作。}\\

\noindent
{\bf 4. (1\%) 畫出CNN model的saliency map,並簡單討論其現象。(ref:\href{ https://reurl.cc/Qpjg8b}{https://reurl.cc/Qpjg8b})}\\

以下為從7種情緒中各隨機挑出一張照片,並使用3-layer CNN model [64, 128, 512] 所畫出的saliency map:\\

\begin{center}
    \includegraphics[width=\textwidth]{saliency_map_1.png}
\end{center}

\begin{center}
    \includegraphics[width=\textwidth]{saliency_map_2.png}\\
\end{center}

由此可以看出,saliency map中數值較大的地方主要是集中在面部五官的部分,因此可以推測CNN主要是透過五官來辨識情緒。\\

\bigskip

\noindent
{\bf 5. (1\%) 畫出最後一層的filters最容易被哪些feature activate。(ref:\\
\href{https://reurl.cc/ZnrgYg}{https://reurl.cc/ZnrgYg})}\\

以下為在訓練完3-layer CNN model [64, 128, 512]之後,從最後一個convolution layer中取出32個filter,找出可以使其最為activate的各個feature:\\

\begin{center}
    \includegraphics[width=\textwidth]{feature_visualization.png}\\
\end{center}

以上的feature看起來都是一些簡單的幾何圖案或材質,比預想中的簡單許多,我猜測其原因為照片中人臉佔了絕大部分的面積,而人的五官本來就是由簡單的幾何形狀所構成,沒有太複雜的特徵,因此只要能偵測簡單的幾何形狀,便有可能辨識出人的五官,進而將照片中的人臉情緒做分類,此外,CNN的架構簡單也有可能為造成該現象的原因之一。\\

\bigskip

\noindent
{\bf \LARGE Part 2. Math Problem}\\

\noindent
{\bf \Large 1. Convolution (1\%)}\\

As we mentioned in class, image size may change after convolution layers. Consider a batch of image data with shape $(B, W, H, input\_channels)$, how will the shape change after the convolution layer?\\

\noindent
$Conv2D (input\_channels,\ output\_channels,\ kernel\_size = (k_1,\ k_2),\ stride = (s_1,\ s_2),\ padding = (p_1,\ p_2))$\\

To simplify the answer:the padding tuple means that we pad $p_1$ pixels on both left and right side, and $p_2$ pixels for top and bottom.\\

\noindent
{\bf \large solution}\\

\noindent
由PyTorch官方API (\href{https://pytorch.org/docs/stable/nn.html#conv2d}{https://pytorch.org/docs/stable/nn.html#conv2d}),可知 $(B, W, H, input\_channels)$ 在經過題幹中所述的convolution layer之後,會變為\\
$(B, \floor*{\frac{W + 2 P_1 - k_1}{s_1} + 1}, \floor*{\frac{H + 2 P_2 - k_2}{s_2} + 1}, output\_channels)$。$\ \square$\\

\bigskip

\noindent
{\bf \Large 2. Batch Normalization (1\%)}\\

Besides Dropout, we usually use Batch Normalization in training nowadays \href{https://arxiv.org/pdf/1502.03167.pdf}{[ref]}. The trick is popular within the deep networks due to its convenience while training. It preserves the distribution within hidden layers and avoids gradient vanish.\\

The algorithm can be written as below:\\

\begin{algorithm}[H]
    \caption{Batch Normalization}
    \SetAlgoLined
    \SetKwInOut{Input}{Input}
    \SetKwInOut{Output}{Output}
    \Input{Values of $x$ over a mini−batch: ${\mathcal B} = \{x_{1 \cdots m}\}$;\\}
    \Output{$\{y_i = BN_{\gamma, \beta}(x_i)\}$;\\
    Parameters to be learned: $\gamma, \beta$;\\}
    $\mu_{\mathcal B} \leftarrow \frac{1}{m} \sum^m_{i = 1} x_i$ //mini−batch mean\\
    $\sigma^2_{\mathcal B} \leftarrow \frac{1}{m} \sum^m_{i = 1} (x_i − \mu_{\mathcal B})^2$ //mini−batch variance\\
    ${\hat x}_i \leftarrow \frac{x_i − \mu_{\mathcal B}}{\sqrt{\sigma^2_{\mathcal B} + \epsilon}}$ //normalize\\
    $y_i \leftarrow \gamma {\hat x}_i + \beta \equiv BN_{\gamma, \beta}(x_i)$ //scale and shift\\
\end{algorithm}

\bigskip

How to update $\gamma$ and $\beta$ from the optimization process of loss?\\

Just try to derive $\frac{\partial l}{\partial {\hat x}_i}, \frac{\partial l}{\partial \sigma^2_{\mathcal B}}, \frac{\partial l}{\partial \mu_{\mathcal B}}, \frac{\partial l}{\partial x_i}, \frac{\partial l}{\partial \gamma}, \frac{\partial l}{\partial \beta}$.\\

\noindent
{\bf \large solution}\\

\noindent
因為
\begin{align*}
    &\mu_{\mathcal B} = \mu_{\mathcal B} (x_1, x_2, \cdots, x_m) = \frac{1}{m} \sum_{i = 1}^m x_i\\
    &\sigma^2_{\mathcal B} = \sigma^2_{\mathcal B} (x_1, x_2, \cdots, x_m, \mu_{\mathcal B}) = \frac{1}{m} \sum_{i = 1}^m (x_i - \mu_{\mathcal B})^2\\
    &{\hat x}_i = {\hat x}_i (x_i, \mu_{\mathcal B}, \sigma^2_{\mathcal B}, \epsilon) = \frac{x_i - \mu_{\mathcal B}}{\sqrt{\sigma^2_{\mathcal B} + \epsilon}}\\
    &y_i = y_i ({\hat x}_i, \gamma, \beta) = \gamma {\hat x}_i + \beta
\end{align*}
所以
\begin{align*}
    \frac{\partial l}{\partial {\hat x}_i} = \frac{\partial l}{\partial y_i} \frac{\partial y_i}{\partial {\hat x}_i} = \frac{\partial l}{\partial y_i} \cdot \frac{\partial}{\partial {\hat x}_i} (\gamma {\hat x}_i + \beta) = \frac{\partial l}{\partial y_i} \cdot \gamma
\end{align*}
\begin{align*}
    \frac{\partial l}{\partial \sigma^2_{\mathcal B}} &= \sum_{i = 1}^m \frac{\partial l}{\partial {\hat x}_i} \frac{\partial {\hat x}_i}{\partial \sigma^2_{\mathcal B}} = \sum_{i = 1}^m \frac{\partial l}{\partial {\hat x}_i} \cdot \frac{\partial}{\partial \sigma^2_{\mathcal B}} \frac{x_i - \mu_{\mathcal B}}{\sqrt{\sigma^2_{\mathcal B} + \epsilon}}\\
    &= \sum_{i = 1}^m \frac{\partial l}{\partial {\hat x}_i} \cdot (x_i - \mu_{\mathcal B}) \cdot \frac{-1}{2} (\sigma^2_{\mathcal B} + \epsilon)^{-\frac{3}{2}}
\end{align*}
\begin{align*}
    \frac{\partial l}{\partial \mu_{\mathcal B}} &= \sum_{i = 1}^m \frac{\partial l}{\partial {\hat x}_i} \frac{\partial {\hat x}_i}{\partial \mu_{\mathcal B}} + \frac{\partial l}{\partial \sigma^2_{\mathcal B}} \frac{\partial \sigma^2_{\mathcal B}}{\partial \mu_{\mathcal B}}\\
    &= \sum_{i = 1}^m \frac{\partial l}{\partial {\hat x}_i} \cdot \frac{\partial}{\partial \mu_{\mathcal B}} \frac{x_i - \mu_{\mathcal B}}{\sqrt{\sigma^2_{\mathcal B}} + \epsilon} + \frac{\partial l}{\partial \sigma^2_{\mathcal B}} \cdot \frac{\partial}{\partial \mu_{\mathcal B}} \frac{1}{m} \sum_{i = 1}^m (x_i - \mu_{\mathcal B})^2\\
    &= \sum_{i = 1}^m \frac{\partial l}{\partial {\hat x}_i} \cdot \frac{-1}{\sqrt{\sigma^2_{\mathcal B} + \epsilon}} + \frac{\partial l}{\partial \sigma^2_{\mathcal B}} \cdot \frac{\sum_{i = 1}^m -2(x_i - \mu_{\mathcal B})}{m}
\end{align*}
\begin{align*}
    \frac{\partial l}{\partial x_i} &= \frac{\partial l}{\partial {\hat x}_i} \frac{\partial {\hat x}_i}{\partial x_i} + \frac{\partial l}{\partial \sigma^2_{\mathcal B}} \frac{\partial \sigma^2_{\mathcal B}}{\partial x_i} + \frac{\partial l}{\partial \mu_{\mathcal B}} \frac{\partial \mu_{\mathcal B}}{\partial x_i}\\
    &= \frac{\partial l}{\partial {\hat x}_i} \cdot \frac{\partial}{\partial x_i} \frac{x_i - \mu_{\mathcal B}}{\sqrt{\sigma^2_{\mathcal B} + \epsilon}} + \frac{\partial l}{\partial \sigma^2_{\mathcal B}} \cdot \frac{\partial}{\partial x_i} \frac{1}{m} \sum_{j = 1}^m (x_j - \mu_{\mathcal B})^2 + \frac{\partial l}{\partial \mu_{\mathcal B}} \cdot \frac{\partial}{\partial x_i} \frac{1}{m} \sum_{j = 1}^m x_j\\
    &= \frac{\partial l}{\partial {\hat x}_i} \cdot \frac{1}{\sqrt{\sigma^2_{\mathcal B} + \epsilon}} + \frac{\partial l}{\partial \sigma^2_{\mathcal B}} \cdot \frac{2(x_i - \mu_{\mathcal B})}{m} + \frac{\partial l}{\partial \mu_{\mathcal B}} \cdot \frac{1}{m}
\end{align*}
\begin{align*}
    \frac{\partial l}{\partial \gamma} = \sum_{i = 1}^m \frac{\partial l}{\partial y_i} \frac{\partial y_i}{\partial \gamma} = \sum_{i = 1}^m \frac{\partial l}{\partial y_i} \cdot \frac{\partial}{\partial \gamma} (\gamma {\hat x}_i + \beta) = \sum_{i = 1}^m \frac{\partial l}{\partial y_i} \cdot \frac{\partial}{\partial \gamma} {\hat x}_i
\end{align*}
\begin{align*}
    \frac{\partial l}{\partial \beta} = \sum_{i = 1}^m \frac{\partial l}{\partial y_i} \frac{\partial y_i}{\partial \beta} = \sum_{i = 1}^m \frac{\partial l}{\partial y_i} \cdot \frac{\partial}{\partial \beta} (\gamma {\hat x}_i + \beta) = \sum_{i = 1}^m \frac{\partial l}{\partial y_i}\ \square\\
\end{align*}

\noindent
{\bf \Large 2. Softmax and Cross Entropy (1\%)}\\

In classification problem, we use softmax as activation function and cross entropy as loss function.

\begin{align*}
    &softmax(z_t) = \frac{e^{z_t}}{\sum_i e^{z_i}}\\
    &cross\_entropy = L(y, {\hat y}) = −\sum_i y_i log{\hat y}_i\\
    &cross\_entropy = L_t(y_t, {\hat y}_t) = −y_t log {\hat y}_t\\
    &{\hat y}_t = softmax(z_t)\\
\end{align*}

Derive that $\frac{\partial L_t}{\partial z_t} = {\hat y}_t - y_t$.\\

\noindent
{\bf \large solution}\\

\noindent
題幹中的 $ L_t(y_t, {\hat y}_t) = −y_t log {\hat y}_t$ 為當 $y_t = 1$ 時的cross entropy loss,此時\\
\begin{align*}
    \frac{\partial L_t}{\partial z_t} &= \frac{\partial L_t}{\partial {\hat y}_t} \frac{\partial {\hat y}_t}{\partial z_t} = \frac{\partial}{\partial {\hat y}_t} (-y_t log {\hat y}_t) \cdot \frac{\partial}{\partial z_t} \frac{e^{z_t}}{\sum_i e^{z_i}}\\
    &= -\frac{y_t}{{\hat y}_t} \cdot \frac{e^{z_t} \cdot \sum_i e^{z_i} - e^{z_t} \cdot e^{z_t}}{(\sum_i e^{z_i})^2}\\
    &= -\frac{y_t}{{\hat y}_t} \cdot \frac{e^{z_t}}{\sum_i e^{z_i}} (1 - \frac{e^{z_t}}{\sum_i e^{z_i}})\\
    &= -\frac{y_t}{{\hat y}_t} \cdot {\hat y}_t (1 - {\hat y}_t) = y_t {\hat y}_t - y_t = {\hat y}_t - y_t
\end{align*}
而當 $y_t = 0$ 時,cross entropy loss應為 $L_t(y_t, {\hat y}_t) = −(1 - y_t) log (1 - {\hat y}_t)$,此時
\begin{align*}
    \frac{\partial L_t}{\partial z_t} &= \frac{\partial L_t}{\partial {\hat y}_t} \frac{\partial {\hat y}_t}{\partial z_t} = \frac{\partial}{\partial {\hat y}_t} (-(1 - y_t) log (1 - {\hat y}_t)) \cdot \frac{\partial}{\partial z_t} \frac{e^{z_t}}{\sum_i e^{z_i}}\\
    &= \frac{1 - y_t}{1 - {\hat y}_t} \cdot \frac{e^{z_t} \cdot \sum_i e^{z_i} - e^{z_t} \cdot e^{z_t}}{(\sum_i e^{z_i})^2}\\
    &= \frac{1 - y_t}{1 - {\hat y}_t} \cdot \frac{e^{z_t}}{\sum_i e^{z_i}} (1 - \frac{e^{z_t}}{\sum_i e^{z_i}})\\
    &= \frac{1 - y_t}{1 - {\hat y}_t} \cdot {\hat y}_t (1 - {\hat y}_t) = {\hat y}_t - y_t {\hat y}_t = {\hat y}_t - y_t\ \square
\end{align*}

\end{document}